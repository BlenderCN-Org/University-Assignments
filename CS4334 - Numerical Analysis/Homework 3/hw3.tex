\documentclass[12pt]{article}

\usepackage[margin=1.0in]{geometry}
\usepackage{tabto}
\usepackage{tikz}
\usepackage{amssymb}
\usepackage{amsmath}
\usepackage{framed}
\usepackage{listings}
\usepackage{color} %red, green, blue, yellow, cyan, magenta, black, white
\definecolor{mygreen}{RGB}{28,172,0} % color values Red, Green, Blue
\definecolor{mylilas}{RGB}{170,55,241}

\newcommand*\circled[1]{\tikz[baseline=(char.base)]{
            \node[shape=circle,draw,inner sep=2pt] (char) {#1};}}

\begin{document}

\title{CS / MATH 4334 : Numerical Analysis\\Homework Assignment 3}
\author{Matthew McMillian\\mgm160130@utdallas.edu}
\maketitle

\lstset{language=Matlab,%
    %basicstyle=\color{red},
    breaklines=true,%
    morekeywords={matlab2tikz},
    morekeywords=[2]{1}, keywordstyle=[2]{\color{black}},
    identifierstyle=\color{black},%
    showstringspaces=false,%without this there will be a symbol in the places where there is a space
    numbers=left,%
    numberstyle={\tiny \color{black}},% size of the numbers
    numbersep=9pt, % this defines how far the numbers are from the text
    emph=[1]{for,end,break},emphstyle=[1]\color{red}, %some words to emphasise
    %emph=[2]{word1,word2}, emphstyle=[2]{style},    
}

\section*{Theoretical Problems}
\pagebreak

\begin{enumerate}

	\item For the MatLab code given,\\a) Compute the exact number of flops (arithmetic operations) in the entire algorithm in terms of
n. You do not need to multiply out the polynomials. \\ \\b) Compute only the highest-power term for the number of flops in the algorithm using
integration. 
	\begin{itemize}
		\item[a)] $\Sigma_{k=1}^{j-1}(2) = 2\Sigma_{k=1}^{j-1}(1) = 2(j-1)$ \\ \\
		$\Sigma_{k=1}^{j-1}(3) = 3\Sigma_{k=1}^{j-1}(1) = 3(j-1)$ \\ \\
		$\Sigma_{i=j+1}^{n}(\Sigma_{k=1}^{j-1}(2)) = 2\Sigma_{i=j+1}^{n}(j-1) = 2\Sigma_{i=j+1}^{n}(j) - 2\Sigma_{i=j+1}^{n}(1) = 2jn-2j^2-2n+2j$ \\ \\
		$\Sigma_{i=j+1}^{n}(\Sigma_{k=1}^{j-1}(2)) + \Sigma_{k=1}^{j-1}(3) = 2jn-2j^2-2n+5j-3$ \\ \\
		$\Sigma_{j=1}^n(\Sigma_{i=j+1}^{n}(\Sigma_{k=1}^{j-1}(2)) + \Sigma_{k=1}^{j-1}(3)) = \Sigma_{j=1}^n(2jn-2j^2-2n+5j-3)$ \\ \\
$\Sigma_{j=1}^n(2jn) - \Sigma_{j=1}^n(2j) - \Sigma_{j=1}^n(2n) + \Sigma_{j=1}^n(5j) - \Sigma_{j=1}^n(3) = \frac{1}{3}n^3 + \frac{5}{2}n^2+\frac{1}{2}n \rightarrow$ \fbox{$O(\frac{1}{3}n^3)$}\\
	\item[b)] $\int_1^n(\int_{j+1}^n(\int_1^{j-1}(2)dk)di + \int_1^{j-1}dk)dj \rightarrow \int_1^n(\int_{j}^n(\int_1^{j}(2)dk)di + \int_1^{j}dk)dj$ \\ \\
	$\int_1^n(\int_{j}^n(2j-2)di + 3j-3)dj \rightarrow \int_1^n(2nj -2n -2j^2 +5j -3)dj$ \\ \\
	$nj^2 - 2nj -\frac{2}{3}j^3 + \frac{5}{2}j^2 3j$ $|_1^n\rightarrow O(\frac{1}{3}n^3 - \frac{1}{2}n^2 -n) \rightarrow $ \fbox{$O(\frac{1}{3}n^3)$}	
	\end{itemize}
	
	\item Find the PA = LU factorization (using partial pivoting) of the matrix A = 
$
\begin{bmatrix}
   1 & 2 & 6 \\
   4 & 8 & -1 \\
   -2 & 3 & 5 \\
\end{bmatrix}
$
Write out P, L, and U for your final answer. \\ \\


A = $\begin{bmatrix}
   1 & 2 & 6 \\
   4 & 8 & -1 \\
   -2 & 3 & 5 \\
\end{bmatrix}$, we pivot our 2nd and 1st row to get A=$\begin{bmatrix}
   4 & 8 & -1 \\
   1 & 2 & 6 \\
   -2 & 3 & 5 \\
\end{bmatrix}$ and \\ \\ \\
P=$\begin{bmatrix}
   0 & 1 & 0\\
   1 & 0 & 0 \\
   0 & 0 & 1 \\
\end{bmatrix}$. Note that for the following, circled values imply our pivots.\\R$_2$ = R$_2$ - $\frac{1}{4}$R$_1$ $\rightarrow$ A=$\begin{bmatrix}
   4 & 8 & -1 \\
   \circled{1/4} & 0 & 25/4 \\
   -2 & 3 & 5 \\
\end{bmatrix}$. 
R$_3$ = R$_3$ + $\frac{1}{2}$R$_1 \rightarrow$ A= $\begin{bmatrix}
   4 & 8 & -1 \\
   \circled{1/4} & 0 & 25/4 \\
   \circled{-1/2} & 7 & 9/2 \\
\end{bmatrix}$.
\pagebreak

We need to perform a row swap of R$_3$ and R$_2$ and we also need to swap our rows in our permutation matrix P. Thus... \\ \\
\fbox{P=$\begin{bmatrix}
    0 & 1 & 0\\
   0 & 0 & 1 \\
   1 & 0 & 0 \\
\end{bmatrix}$, L = $\begin{bmatrix}
    1 & 0 & 0\\
   -1/2 & 1 & 0 \\
    1/4 & 0 & 1 \\
\end{bmatrix}$, U = $\begin{bmatrix}
    4 & 8 & -1\\
   0 & 7 & 9/2 \\
   0 & 0 & 25/4 \\
\end{bmatrix}$
} \\

\item Let A=$\begin{bmatrix}
    -1 & 0 & 2\\
   3 & -2 & 1 \\
   -3 & 1 & 0 \\
\end{bmatrix}$. Verify properties (i) and (ii) of the matrix 1-norm are satisfied for A. Now, define another matrix, B$\begin{bmatrix}
    3 & -2 & 3\\
   1 & 0 & 1 \\
   2 & 0 & -2 \\
\end{bmatrix}$. Verify that property (iii) is satisfied for both A and B. \\ \\
The 1-norm is the max of the absolute value of the summation of the elements of columns of a matrix. \\ \\
(i) $||$A$||$ $\geq$ 0 iff A is nonzero. \\
$||$A$||$ = max$_{1\leq j\leq n}$($\Sigma_{j=1}^n |a_{ij}|$). The max column sum is column 1 with a sum of 7 $\neq$ 0, therefore the first property holds. \\ \\
(ii) $||\alpha$A$||$ = $|\alpha|$ $||$A$||$. \\
Assume we multiply every term within A by $\alpha$, such that every term in A is the same term followed (multiplied) by some $\alpha$. If we take our 1-norm again, we obtain $||\alpha$A$||$ = $|7\alpha|$. Trivially this shows that $||\alpha$A$||$ = $|\alpha|$ $||$A$||$, therefore the second property holds. \\ \\
(iii) $||$A+B$||$ $\leq$ $||$A$||$ + $||$B$||$. \\ \\
Consider A+B = $\begin{bmatrix}
    2 & -2 & 5\\
   4 & -2 & 2 \\
   -1 & 1 & -2 \\
\end{bmatrix}$. $||$A$||$ = 7. $||$B$||$ = 6. $||$A+B$||$ = 9.\\ \\ Plugging it into the inequality, 9 $\leq$ 7 + 6 $\rightarrow$ 9 $\leq$ 13 which is true, so property 3 holds. This can be see for matrix B as well due to the commutative nature of addition.\\
\pagebreak
\item Consider the following linear system: \\ \\
$\begin{bmatrix}
    2 & -1-\epsilon\\
   2 & -1 \\
\end{bmatrix}$ $\begin{bmatrix}
   x_1 \\
   x_2 \\
\end{bmatrix}$ = $\begin{bmatrix}
   1-\epsilon \\
   1 \\
\end{bmatrix}$, where $x_c$ is $\begin{bmatrix}
   1+\epsilon \\
   1 \\
\end{bmatrix}$, and 0 $<$ $\epsilon$ $<<$ 1. \\ \\
Use the infinity norm to find the condition number, the relative backward error, an upper bound on the forward solution, the relative forward error, and the limit of the condition number as $\epsilon$ approaches 0.
	\begin{enumerate}
		\item cond(A) = $||$A$||$ $||$A$^{-1}||$ \\
		A$^{-1}$ = $\frac{1}{det(A)} \begin{bmatrix}
    -1 & 1+\epsilon\\
   -2 & 2 \\
\end{bmatrix}$ =  $\frac{1}{2\epsilon} \begin{bmatrix}
    -1 & 1+\epsilon\\
   -2 & 2 \\
\end{bmatrix}$ \\
$||$A$||_{\infty}$ = 3 + $\epsilon$ ; $||$A$^{-1}||$ = $\frac{3}{2\epsilon}$ \\
Thus, $||$A$||$ $||$A$^{-1}||$ = \fbox{$\cfrac{3(3+\epsilon)}{2\epsilon}$}.
\item Relative Backward Error = $\cfrac{||Ax - A\bar{x}||_{\infty}}{||\bar{b}||_{\infty}}$. = $\cfrac{1-2\epsilon}{1}$ = \fbox{$1-2\epsilon$}.
\item Upper Bound of Relative Forward Error is cond(A)*relbackerr = $\frac{3(3+\epsilon)(1-2\epsilon)}{2\epsilon}$. At $\epsilon$ = 0.001, our forward error upper bound is \fbox{4492.497}
\item Relative Forward Error = $\cfrac{||x - x_c||_{\infty}}{||x||_{\infty}}$ = $|$ 1 - (1+$\epsilon$) $|$ = \fbox{$\epsilon$}. At $\epsilon$ = 0.001, our forward error is \fbox{0.001}.
\item $lim_{\epsilon \rightarrow 0}$ cond(A) $\rightarrow \infty$. When $\epsilon$ is 0, the cond(A) is undefined since you cannot divide by 0, which means that our A matrix is VERY ill-conditioned. \\
	\end{enumerate}
\item Consider the system given below. Verify that the matrix is diagonally dominant. Then compute the first two steps of Gauss-Seidel Method starting with a 0 vector. \\ \\
$\begin{bmatrix}
   3 & -1 & 1 \\
   3 & 6 & 2 \\
   3 & 3 & 7 \\
\end{bmatrix}$ \tabto{2.5cm} Checking Row 1: 3 $>$ ($|$-1$|$ + $|$1$|$). Checking Row 2: 6 $>$ ($|$3$|$ + $|$2$|$). 
\tabto{2.5cm} Checking Row 3: 7 $>$ ($|$3$|$ + $|$3$|$). \\
\fbox{Therefore, the rows are diagonally dominant}. \\ \\
After calculations, we can determine the following equations: \\
x = $\frac{1}{3}(1-z+y)$, y = $\frac{1}{6}(-2z-3x)$, z = $\frac{1}{7}(4-3x-3y)$\\
Thus we get D=$\begin{bmatrix}
   1/3 & 0 & 0 \\
   0 & 1/6 & 0 \\
   0 & 0 & 1/7 \\
\end{bmatrix}$, L=$\begin{bmatrix}
   0 & 0 & 0 \\
   -3 & 0 & 0 \\
   -3 & -3 & 0 \\
\end{bmatrix}$, and U=$\begin{bmatrix}
   0 & -1 & 1 \\
   0 & 0 & -2 \\
   0 & 0 & 0 \\
\end{bmatrix}$, $\bar{b}$=$\begin{bmatrix}
   1\\
   0\\
   4\\
\end{bmatrix}$ \pagebreak

Our final Gauss-Seidel Method: $x_{n+1} = (D-L)^{-1}[U\bar{x}_n + \bar{b}]$\\ \\
$x_{n+1} = \begin{bmatrix}
   1/3 & 0 & 0 \\
   -1/6 & 1/6 & 0 \\
   -1/14 & -1/14 & 1/7 \\
\end{bmatrix}$ ($\begin{bmatrix}
   0 & -1 & 1 \\
   0 & 0 & -2 \\
   0 & 0 & 0 \\
\end{bmatrix}$ $\bar{x}_n$ + $\begin{bmatrix}
   1 \\
   0   \\
   4   \\
\end{bmatrix}$) \\ \\ \\
Thus we get \fbox{$x_1$ = $\begin{bmatrix}
   1/3 \\
   -1/6   \\
   1/2   \\
\end{bmatrix}$, and $x_2$ = $\begin{bmatrix}
   5/9 \\
   -4/9   \\
   11/21   \\
\end{bmatrix}$} as our first two iterations.


\end{enumerate}
\end{document}