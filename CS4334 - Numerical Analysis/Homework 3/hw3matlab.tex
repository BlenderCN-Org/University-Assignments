\documentclass[12pt]{article}

\usepackage[margin=1.0in]{geometry}
\usepackage{tabto}
\usepackage{amsmath}
\usepackage{framed}
\usepackage{listings}
\usepackage{color} %red, green, blue, yellow, cyan, magenta, black, white
\definecolor{mygreen}{RGB}{28,172,0} % color values Red, Green, Blue
\definecolor{mylilas}{RGB}{170,55,241}

\newcommand*\lstinputpath[1]{\lstset{inputpath=#1}}



\begin{document}

\title{CS / MATH 4334 : Numerical Analysis\\Homework Assignment 3}
\author{Matthew McMillian\\mgm160130@utdallas.edu}
\maketitle

\section*{MatLab Problems}

\lstset{language=Matlab,%
    %basicstyle=\color{red},
    breaklines=true,%
    morekeywords={matlab2tikz},
    morekeywords=[2]{1}, keywordstyle=[2]{\color{black}},
    identifierstyle=\color{black},%
    showstringspaces=false,%without this there will be a symbol in the places where there is a space
    numbers=left,%
    numberstyle={\tiny \color{black}},% size of the numbers
    numbersep=9pt, % this defines how far the numbers are from the text
    emph=[1]{for,end,break},emphstyle=[1]\color{red}, %some words to emphasise
    %emph=[2]{word1,word2}, emphstyle=[2]{style},    
}

\pagebreak

	\begin{enumerate}
	
	\item[] Problem 1 : gaelpp.m \noindent\rule{\textwidth}{1.0pt} \\
	\lstinputpath{P1}
	\lstinputlisting{gaelpp.m}
	
	\item[] Problem 1 : gaelppscript.m \noindent\rule{\textwidth}{1.0pt} \\
	\lstinputlisting{gaelppscript.m}	
	
	\item[] Problem 1 : gaelscript.m \noindent\rule{\textwidth}{1.0pt} \\
	\lstinputlisting{gaelscript.m}	
	
	\pagebreak	
	
	$>>$ gaelppscript.m + gaelscript.m
	\begin{framed}
	(In gaelppscript.m)\\
	A =\\
     6.110000000000000e+00    -1.420000000000000e+01     2.100000000000000e+01\\
    -4.959083469721767e-01     5.058101472995091e+00     3.414075286415711e+00\\
     4.959083469721767e-01    -1.000000000000000e+00     7.000000000000000e+00\\


	P =\\
     0     0     1\\
     0     1     0\\
     1     0     0\\


	y (forsub) =\\
    -1.190000000000000e+02\\
     6.098690671031097e+01\\
    -1.899999999999999e+01\\


	s (backsub) =\\
     2.213234104513831e+01\\
     1.388933829477430e+01\\
    -2.714285714285713e+00\\
    
   (In gaelscript.m)\\ 
    x =\\
                       Inf\\
                       Inf\\
    -1.396257416704701e+01\\


ans =\\
   NaN\\
   NaN\\
   NaN\\

We can tell from our answer that partial pivoting helps avoid swamping errors that regular Gaussian Elimination cannot deal with.
	\end{framed}
	
	\end{enumerate}
	
\pagebreak	
	
	\begin{enumerate}
	
	\item[] Problem 2 : hw2.m \noindent\rule{\textwidth}{1.0pt} \\
	\lstinputpath{P2}
	\lstinputlisting{hw2.m}	
	
	\pagebreak	
	
	$>>$ hw2.m
	\begin{framed}
relBkwdErr =\\
     6.329171661699566e-17\\

condA =\\
     2.106245945721575e+12\\


boundOnFwrdError =\\
     1.333079215223059e-04\\


relFwrdErr =\\
     6.074920350417704e-06\\

Since we are using double percision, our answer is only accurate to 10$^{16}$ - cond(A) digits = 16 - log$_{10}$(2.106245945721575e+12)  $\approx$ 3.67, so we can trust $\approx$ 3 digits.
	\end{framed}
	
	\end{enumerate}
	
\pagebreak	
	
	\begin{enumerate}
	
	\item[] Problem 3 : newt3d.m \noindent\rule{\textwidth}{1.0pt} \\
	\lstinputpath{P3}
	\lstinputlisting{newt3d.m}	
	
	\item[] Problem 3 : genmatrix.m \noindent\rule{\textwidth}{1.0pt} \\
	\lstinputlisting{genmatrix.m}	
	
	\item[] Problem 3 : genrhs.m \noindent\rule{\textwidth}{1.0pt} \\
	\lstinputlisting{genrhs.m}	
	
	\pagebreak	
	
	$>>$ newt3d.m
	\begin{framed}
	x =\\
     8.771286446121147e+00\\
     2.596954489674528e-01\\
    -1.372281323269016e+00\\

ans =\\
     3.901377355679192e-07 \\ 
     
     Note that if we do not log-transform the equations before hand we will have an ill-conditioned matrix.
	\end{framed}
	
	\end{enumerate}
	
\end{document}