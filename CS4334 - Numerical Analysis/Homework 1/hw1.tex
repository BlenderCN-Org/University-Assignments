\documentclass[12pt]{article}

\usepackage[margin=1.0in]{geometry}
\usepackage{tabto}
\usepackage{amsmath}

\begin{document}

\title{CS / MATH 4334 : Numerical Analysis\\Homework Assignment 1}
\author{Matthew McMillian\\mgm160130@utdallas.edu}
\maketitle

\section*{Theoretical Problems}

\begin{enumerate}

	\item Show how to evaluate the polynomial \(p(x) = 2x^{25} + 7x^{15} - x^{10} + 4x^{5} - 1\) using as few arithmetic operations as possible. \\ \\
	To begin to solve this problem, we must first break down the polynomial p(x). I will start by storing some of the possible polynomial values ahead of time to conserve arithmetic operations.
	
	\begin{itemize}
		\item[] \(x = x\) \tabto{6cm} - identity \tabto{10cm} (0+0 = 0 multiplications)
		\item[] \(x^2 = x * x\) \tabto{6cm} - 1 multiplication  \tabto{10cm} (0+1 = 1 multiplications)
		\item[] \(x^4 = x^2 * x^2\) \tabto{6cm} - 1 multiplication \tabto{10cm} (1+1 = 2 multiplications)
		\item[] \(x^5 = x^4 * x\) \tabto{6cm} - 1 multiplication \tabto{10cm} (2+1 = 3 multiplications)
	\end{itemize}
	
In total, storing values until we have stored an \(x^5\) will net us a total of 3 multiplications. Next, we will apply a Horner's expansion to the function \(p(x)\), given by:	
	
	\begin{itemize}
		\item[] \(p(x) = 2x^{25} + 7x^{15} - x^{10} + 4x^{5} - 1\)
		\item[] \(p(x) = x^5 * (2x^{20}  +7x^{10} - x^5 + 4) - 1\)
		\item[] \(p(x) = x^5 * ( x^5 * ( 2x^{15} + 7x^5 - 1) + 4) -1\)
		\item[] \(p(x) = x^5 * (x^5 * (x^5 * (2x^{10} + 7 ) - 1 ) +4 ) - 1\)
		\item[] \(p(x) = x^5 * (x^5 * (x^5 *( x^5 *(2x^5 + 0) +7) -1)+4)-1\)
		\item[] \(p(x) = x^5 * (x^5 * (x^5 *( x^5 *(x^5 * (2) + 0) +7) -1)+4)-1\)
	\end{itemize}
		
Expanding \(p(x)\) with Horner's method and storing the calculations prior to the functions evaluation nets us a total of \textbf{8 multiplications} and \textbf{5 additions or subtractions}.\\ \\ \\

	\item Convert the binary number 101101.000$\overline{1011}$ to decimal form. \\
	
	To convert this binary number to a decimal number, we must convert the items on both the \textsc{left} and \textsc{right} of the radix point. From the \textsc{left}, we use the algorithm to multiply the binary digits by their respective \(2^x\) compliments: 
	
	\begin{itemize}
		\item[] 101101$_2$ = $(1*2^5) + (0*2^4) + (1*2^3) + (1*2^2) + (0*2^1) + (1*2^0)$ = 45$_{10}$
	\end{itemize}
	
From the \textsc{right}, we could normally multiply by a base 2 number algorithm. However, the repeating decimal makes it tricky. In this case, we must normalize our decimal number and apply some algebra tricks to obtain and approximate value for the decimal.

	\begin{itemize}
		\item[] $x = .000\overline{1011}$
		\item[] $2^3x = .\overline{1011}$
		\item[] $2^7x = 1011.\overline{1011}$
		\item[] $2^7x - 2^3x = (1011)_2 = (11)_{10}$
		\item[] $x*(2^7 - 2^3) = 111$
		\item[] $x*(120)= 11$
		\item[] $x = \frac{11}{120} \approx .091\overline{6}$
	\end{itemize}
	
Thus our final base 10 decimal number is a concatenation of our \textsc{left} and \textsc{right} numbers, which is \textbf{$45.091\overline{6}$}.\\

	\item Consider the decimal number $-26.1$. Convert this number to binary form, then determine the machine representation of this number in double precision. Give the entire set in hexadecimal form. \\
	
	

\end{enumerate}

\end{document}