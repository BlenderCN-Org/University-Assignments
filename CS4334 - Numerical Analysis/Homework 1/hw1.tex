\documentclass[12pt]{article}

\usepackage[margin=1.0in]{geometry}
\usepackage{tabto}
\usepackage{amsmath}
\usepackage{framed}
\usepackage{listings}
\usepackage{color} %red, green, blue, yellow, cyan, magenta, black, white
\definecolor{mygreen}{RGB}{28,172,0} % color values Red, Green, Blue
\definecolor{mylilas}{RGB}{170,55,241}

\begin{document}

\title{CS / MATH 4334 : Numerical Analysis\\Homework Assignment 1}
\author{Matthew McMillian\\mgm160130@utdallas.edu}
\maketitle

\lstset{language=Matlab,%
    %basicstyle=\color{red},
    breaklines=true,%
    morekeywords={matlab2tikz},
    morekeywords=[2]{1}, keywordstyle=[2]{\color{black}},
    identifierstyle=\color{black},%
    showstringspaces=false,%without this there will be a symbol in the places where there is a space
    numbers=left,%
    numberstyle={\tiny \color{black}},% size of the numbers
    numbersep=9pt, % this defines how far the numbers are from the text
    emph=[1]{for,end,break},emphstyle=[1]\color{red}, %some words to emphasise
    %emph=[2]{word1,word2}, emphstyle=[2]{style},    
}

\section*{Theoretical Problems}

\begin{enumerate}

	\item Show how to evaluate the polynomial \(p(x) = 2x^{25} + 7x^{15} - x^{10} + 4x^{5} - 1\) using as few arithmetic operations as possible. \\ \\
	To begin to solve this problem, we must first break down the polynomial p(x). I will start by storing some of the possible polynomial values ahead of time to conserve arithmetic operations.
	
	\begin{itemize}
		\item[] \(x = x\) \tabto{6cm} - identity \tabto{10cm} (0+0 = 0 multiplications)
		\item[] \(x^2 = x * x\) \tabto{6cm} - 1 multiplication  \tabto{10cm} (0+1 = 1 multiplications)
		\item[] \(x^4 = x^2 * x^2\) \tabto{6cm} - 1 multiplication \tabto{10cm} (1+1 = 2 multiplications)
		\item[] \(x^5 = x^4 * x\) \tabto{6cm} - 1 multiplication \tabto{10cm} (2+1 = 3 multiplications)
	\end{itemize}
	
In total, storing values until we have stored an \(x^5\) will net us a total of 3 multiplications. Next, we will apply a Horner's expansion to the function \(p(x)\), given by:	
	
	\begin{itemize}
		\item[] \(p(x) = 2x^{25} + 7x^{15} - x^{10} + 4x^{5} - 1\)
		\item[] \(p(x) = x^5 * (2x^{20}  +7x^{10} - x^5 + 4) - 1\)
		\item[] \(p(x) = x^5 * ( x^5 * ( 2x^{15} + 7x^5 - 1) + 4) -1\)
		\item[] \(p(x) = x^5 * (x^5 * (x^5 * (2x^{10} + 7 ) - 1 ) +4 ) - 1\)
		\item[] \(p(x) = x^5 * (x^5 * (x^5 *( x^5 *(2x^5 + 0) +7) -1)+4)-1\)
		\item[] \(p(x) = x^5 * (x^5 * (x^5 *( x^5 *(x^5 * (2) + 0) +7) -1)+4)-1\)
	\end{itemize}
		
Expanding \(p(x)\) with Horner's method and storing the calculations prior to the functions evaluation nets us a total of \fbox{8 multiplications and 5 additions or subtractions}.\\ \\ \\

	\item Convert the binary number 101101.000$\overline{1011}$ to decimal form. \\
	
	To convert this binary number to a decimal number, we must convert the items on both the \textsc{left} and \textsc{right} of the radix point. From the \textsc{left}, we use the algorithm to multiply the binary digits by their respective \(2^x\) compliments: 
	
	\begin{itemize}
		\item[] 101101$_2$ = $(1*2^5) + (0*2^4) + (1*2^3) + (1*2^2) + (0*2^1) + (1*2^0)$ = 45$_{10}$
	\end{itemize}
	
From the \textsc{right}, we could normally multiply by a base 2 number algorithm. However, the repeating decimal makes it tricky. In this case, we must normalize our decimal number and apply some algebra tricks to obtain and approximate value for the decimal.

	\begin{itemize}
		\item[] $x = .000\overline{1011}$
		\item[] $2^3x = .\overline{1011}$
		\item[] $2^7x = 1011.\overline{1011}$
		\item[] $2^7x - 2^3x = (1011)_2 = (11)_{10}$
		\item[] $x*(2^7 - 2^3) = 111$
		\item[] $x*(120)= 11$
		\item[] $x = \frac{11}{120} \approx .091\overline{6}$
	\end{itemize}
	
Thus our final base 10 decimal number is a concatenation of our \textsc{left} and \textsc{right} numbers, which is \fbox{$45.091\overline{6}$}.\\

	\item Consider the decimal number $-26.1$. Convert this number to binary form, then determine the machine representation of this number in double precision. Give the entire set in hexadecimal form. \\
	
	To convert this decimal number to binary, we must convert both sides of the radix point using the divide and multiply algorithm:
	
	\begin{center}
	\begin{tabular}{ c c }
	Left of Radix Point   & Right of Radix Point  \\
	$26\div2 = 13r0$ & $.1 * 2 = .2 r0$  \\ 
 	$13\div2 = 6r1$  & $.2 * 2 = .4 r0$  \\  
 	$6\div2 = 3r0$   & $.4 * 2 = .8 r0$  \\
 	$3\div2 = 1r1$   & $.8 * 2 = 1.6 r1$ \\
 	$1\div2 = 0r1$   & $.6 * 2 = 1.2 r1$ \\
 					 & $.2 * 2 = 4 r0$ \\
 	 				 & continuing... \\
 					 
	\end{tabular}
	\end{center}

Thus given by applying the algorithm, we obtain a binary number of -11010.0$\overline{0011}$ for part (a). Now, to determine the machine representation of this binary number in double precision, we must find out 3 things; the sign bit, the exponent bits, and the mantissa. 

	\begin{itemize}
		\item The sign bit is easy. Since we have a negative number our sign bit is set to 1.
		\item For the exponent, we must first normalize the binary number, and add an offset of 1023 to our exponent to obtain the correct number for the computer:
			\begin{itemize}
				\item[] Given, e = 11010.0$\overline{0011} * 2^0$
				\item[] Then we normalize, e = 1.10100$\overline{0011} * 2^4$
				\item[] Then we add an offset to our exponent, 1023 + 4 = 1027 
				\item[] Then we convert to binary, $(1027)_{10} = (10000000011)_2$
			\end{itemize}
		
		\item For the mantissa, we put in the numbers after our normalized radix point and check for rounding:
			\begin{itemize}
				\item[] Let $m$ be the mantissa
				\item[] Given $m$ = $|_{0}$ 1010 $|$ 0001 $|$ 1001 $|$ 1001 $|$ 1001 $|$ ... $|$ 1001 $_{52}|_{53}$ 1001 $|$ ...
				\item[] Then we round $m$. Since the $53^{nd}$ bit is a 1, and there are non-zero bits after it, we add a 1 to the $52^{nd}$ bit.
				\item[] Thus our $m$ = $|_{0}$ 1010 $|$ 0001 $|$ 1001 $|$ 1001 $|$ 1001 $|$ ... $|$ 1010 $_{52}|$
			\end{itemize}
			
		\item To finalize the problem, we must concatenate our sign bit, exponent bits, and mantissa bits and convert the entire double precision expression to hexadecimal: 
			\begin{itemize}
				\item[] Let $s, e, m$ = the sign bit, exponent bits, and mantissa respectively.
				\item[] Let $d$ = the concatenation of $s, e, $ and $m$, equivalently $d$ = $s + e + m$.
				\item[] $d$ = 1100$|$0000$|$0011$|$1010$|$0001$|$1001$|$1001$|$1001$|$1001$|$1001$|$1001$|$1001$|$1001$|$1001$|$1010
				\item[] Converting to hexadecimal, $d$ = C03A1999999999A
			\end{itemize}
			
	\end{itemize}

Thus, our final binary number and double precision hexadecimal number are \fbox{-11010.0$\overline{0011}$} and \fbox{C03A1999999999A} respectively.\\

	\item Suppose a certain computer stores decimal numbers (instead of binary) by chopping (instead of rounding) each normalized decimal number to 6 significant digits, i.e., $d_0.d_1d_2d_3d_4d_5$, where $d_0 \neq 0$. Find an upper bound on the relative error from this chopping. \\ 
	
	To find the relative error for this computer, we define relative error to be $|\cfrac{x -x_c}{x}|$, where $x, x_c$ are an exact number and an approximate number respectively.	
		\begin{itemize}
			\item[] Let $x = d_0.d_1d_2d_3d_4d_5 * 10^p$ be exact.
			\item[] Let $x_c = fl(x) = d_0.d_1d_2d_3d_4d_5 * 10^p$ be approximate.
			\item[] Then, our relative error becomes $|\cfrac{x-x_c}{x}|$ = $|\cfrac{0.d_1d_2d_3d_4d_5}{9.d_1d_2d_3d_4d_5}|$.
			\item[] Then, we try to find an upper bound to our error by finding the maximum value of the numerator $n_{max}$, and minimum value of the denominator $d_{min}$, which are $n_{max}$ = 0.00009 and $d_{min}$ = 9.00000.
			\item[] This results in the relative error of  $|\cfrac{0.00009}{9}| = 0.00001 = 10^{-6}$.
		\end{itemize}
		Therefore, similarly to the example in class, we determine that the upper bound of relative error for chopping roundoff error is \fbox{$10^{-6}$}, or more generally, $10^{-p}$, where p is the significant digits of your system.\\
		
		\item Show how roundoff error can propagate through multiplication, given $x_c, y_c$ are approximate values and $x,y$ are exact values. Also, let $x_c = x + \epsilon$ and $y_c = y + \epsilon$. \\
	
	To find the roundoff error propagation, we must substitute in the values for $x_c$ and $y_c$ and solve to find a representation that has an error of $\frac{1}{2} \epsilon_{mach}$.
		\begin{itemize}
			\item[] Given \tabto{1.5cm} $|\cfrac{x_cy_c - xy}{xy}|$ = $|\cfrac{(x + \epsilon_x)(y + \epsilon_y) - xy}{xy}|$ = $|\cfrac{xy + x\epsilon_y + y\epsilon_x + \epsilon_x\epsilon_y - xy}{xy}|$ = \\
			\item[] \tabto{1.5cm} $|\cfrac{x\epsilon_y + y\epsilon_x + \epsilon_x\epsilon_y}{xy}|$ = $|\cfrac{\epsilon_y}{y} + \cfrac{\epsilon_x}{x} + \cfrac{\epsilon_x\epsilon_y}{xy}|$ = $|\cfrac{\epsilon_y}{y}| + |\cfrac{\epsilon_x}{x}| + |\cfrac{\epsilon_x\epsilon_y}{xy}|$ = \\
			\item[] \tabto{1.5cm} $2^{-53} + 2^{-53} + (2^{-53})^2 \approx 2^{-52} + 2^{-106} \leq \frac{1}{2}\epsilon_{mach}$ = $2^{-53}$
		\end{itemize}
		
	Thus we have proven that the multiplication roundoff error has a bounded error of \fbox{$2^{-52} + 2^{-106}$} $\leq \frac{1}{2}\epsilon_{mach}$. \\

	\item Let $x = 12345$, $y = 777.76$, and $z = 0.000321123$. Compute $(x-y)*z$ using floating point arithmetic while rounding to 4 significant digits with the usual rounding rules. \\
	We evaluate this expression using PEMDAS rules:
	\begin{itemize}
		\item[] $(x-y)$: 
			\begin{itemize} 
				\item[] Given $x = 12345$ and $y = 777.76$, we normalize each variable.
				\item[] $x_{norm} = 1.235 * 10^4$, $y_{norm} = 7.778 * 10^2$.
				\item[] $(x_{norm} - y_{norm})$ = $(1.245*10^4 - 7.778*10^2) = 1.157*10^4 = \gamma$		
			\end{itemize}	
		\item[] $(\gamma)*z$:
			\begin{itemize} 
				\item[] Given $\gamma = 1.157*10^4$ and $z = 0.0003213$
				\item[] $\gamma = \gamma_{norm} = 1.158*10^4$, $z_{norm} = 0.000 * 10^4$
				\item[] $(\gamma_{norm}) * (z_{norm}) = (1.157 * 10^4 * 0.000*10^4) = 0.000$.
			\end{itemize}	
	\end{itemize}

Thus, our resulting answer of $(x-y)*z = \fbox{0.000}$. \\

	\item Consider the function $f(x) = \cfrac{cos(2x) - 1}{2x^2}$. Explain why it is problematic for computational math of x near certain values, use algebra or trigonometric identities to fix the function, then find the first three terms of an approximation of $cos(x) = \Sigma^{\infty}_{k=0} \frac{(-1)^kx^{2k}}{(2k)!}$. \\
	
	a.) The original function of $f(x)$ is problematic for specific values of $x$ since it can introduce loss of significance. Consider if we get an $x$ value such that $cos(x)$ is extremely close to 1, but not exactly 1. This would introduce \fbox{loss of significance} since we could lose important bit information due to rounding in the subtraction operation. \\ \\
	b.) To fix this error, we can substitute $cos(x)$ with a trig identity equivalent to get rid of the $-1$.
	\begin{itemize}
		\item[] By trig identity, $cos(x) \equiv 1 - 2sin^2(x)$.
		\item[] Substituting this in, we obtain $f(x) = \cfrac{1-2sin^2(x) - 1}{2x^2}$ = \fbox{$\cfrac{-sin^2(x)}{x^2}$}.
	\end{itemize}

	We have eliminated the subtraction operator in our equation, thus we can now safely use $f(x)$ computationally. \\
	
	c.) To find an approximation of $f(x)$ that is less error prone, we can find the first 3 non-zero terms of the Maclaurin series $cos(x) = \Sigma^{\infty}_{k=0} \frac{(-1)^kx^{2k}}{(2k)!}$.
	\begin{itemize}
		\item[] $k=0:$ \tabto{2cm} $\cfrac{(-1)^0x^{(2*0)}}{(2*0)!}$ = 1.\\
		\item[] $k=1:$ \tabto{2cm} $\cfrac{(-1)^1x^{(2*1)}}{(2*1)!}$ = $\cfrac{-x^2}{2}$.\\
		\item[] $k=2:$ \tabto{2cm} $\cfrac{(-1)^2x^{(2*2)}}{(2*2)!}$ = $\cfrac{x^4}{24}$.
	\end{itemize}
	
	Thus, assuming $x \neq 0$, the first 3 terms are the first non-zero terms of $f(x)$ are \fbox{$1$, $\cfrac{-x^2}{2}$, and $\cfrac{x^4}{24}$}.
	\\ \\ \\ \\ \\ \\ \\ \\ \\ \\ \\ \\ \\ \\ \\
\end{enumerate} 

\section*{MatLab Problems}

	\begin{enumerate}
	
	\item[] Problem 1 : nest.m \noindent\rule{\textwidth}{1.0pt} \\
	\lstinputlisting{nest.m}
	
	\item[] Problem 1 : nest-script.m \noindent\rule{\textwidth}{1.0pt} \\
	\lstinputlisting{hw1_1.m}	
	
	$>>$ nest-script.m 
	\begin{framed}
	Evaluation of p(-2): -67339393.000000
	\end{framed}
	
	\end{enumerate}
	
\pagebreak	
	
	\begin{enumerate}
	\item[] Problem 2 : arctanseries.m \noindent\rule{\textwidth}{1.0pt} \\
	\lstinputlisting{arctanseries.m}	
	
	\item[] Problem 2 : expseries-script.m \noindent\rule{\textwidth}{1.0pt} \\
	\lstinputlisting{hw1_2.m}	

	$>>$ expseries-script.m 
	\begin{framed}
a)\\
  \tabto{0.5cm}1.a) Actual. value of pi from MatLab = 3.141593e+00\\
   \tabto{0.5cm}1.b) Approx. value of pi using Maclaurin series: 4arctan(1) = 3.141597e+00\\
   \tabto{0.5cm}2) Absolute Error: 4.140539e-06\\
   \tabto{0.5cm}3) Relative Error: 1.317974e-06\\
   \tabto{0.5cm}4) Number of 'k' terms needed to approx. in single percision: 16777216\\
b)\\
   \tabto{0.5cm}Eventually the next value in the series becomes extremely small (since k is constantly increasing in the denominator, the next value to be added in the series will be small). This is the result of the phenomenon SWAMPING, since we are trying to add two numbers whose sizes are very different (one large and one extremely small). Therefore, the percision will eventually lose track of the very small values computed due to the rounding and computational limitations.	
	\end{framed}
	\end{enumerate}

	\pagebreak

	\begin{enumerate}
	\item[] Problem 3 : quadroots.m \noindent\rule{\textwidth}{1.0pt} \\
	\lstinputlisting{quadroots.m}	
	
		\item[] Problem 3 : quadscript.m \noindent\rule{\textwidth}{1.0pt} \\
	\lstinputlisting{hw1_3.m}	
		
	$>>$ quadscript.m
	\begin{framed}
		Function Roots (Computational):\\
function1(x1): 9.999999999000000e+04\\
function1(x2): 1.000000000100000e-05\\
function2(x1): -1.000000000100000e-05\\
function2(x2): -9.999999999000000e+04\\
\\
Function Actuals (With Computational Roots):\\
function1(x1): 0.000000000000000e+00\\
function1(x2): 0.000000000000000e+00\\
function2(x1): 0.000000000000000e+00\\
function2(x2): 0.000000000000000e+00\\
\\
Function Roots (Non-Computational):\\
function1(x1): 9.999999999000000e+04\\
function1(x2): 1.000000338535756e-05\\
function2(x1): -1.000000338535756e-05\\
function2(x2): -9.999999999000000e+04\\
\\
Function Actuals (Backwards Errors With Non-Computational Roots):\\
function1(x1): 0.000000000000000e+00\\
function1(x2): 3.384357558644524e-07\\
function2(x1): 3.384357558644524e-07\\
function2(x2): 0.000000000000000e+00\\
	\end{framed}	
	
	\end{enumerate}

\end{document}