\documentclass[12pt]{article}

\usepackage[margin=1.0in]{geometry}
\usepackage{tabto}
\usepackage{tikz}
\usepackage{amssymb}
\usepackage{amsmath}
\usepackage{framed}
\usepackage{listings}
\usepackage{array}% http://ctan.org/pkg/array
\usepackage{color} %red, green, blue, yellow, cyan, magenta, black, white
\definecolor{mygreen}{RGB}{28,172,0} % color values Red, Green, Blue
\definecolor{mylilas}{RGB}{170,55,241}

\newcommand*\circled[1]{\tikz[baseline=(char.base)]{
            \node[shape=circle,draw,inner sep=2pt] (char) {#1};}}

\begin{document}

\title{CS / MATH 4334 : Numerical Analysis\\Homework Assignment 4}
\author{Matthew McMillian\\mgm160130@utdallas.edu}
\maketitle

\lstset{language=Matlab,%
    %basicstyle=\color{red},
    breaklines=true,%
    morekeywords={matlab2tikz},
    morekeywords=[2]{1}, keywordstyle=[2]{\color{black}},
    identifierstyle=\color{black},%
    showstringspaces=false,%without this there will be a symbol in the places where there is a space
    numbers=left,%
    numberstyle={\tiny \color{black}},% size of the numbers
    numbersep=9pt, % this defines how far the numbers are from the text
    emph=[1]{for,end,break},emphstyle=[1]\color{red}, %some words to emphasise
    %emph=[2]{word1,word2}, emphstyle=[2]{style},    
}

\section*{Theoretical Problems}
\pagebreak

\begin{enumerate}
	\item For the 4 points, (-2,0), (-1,2), (1,-3), and (3,1), a) Set up the Lagrange form of the interpolating polynomial. You do not need to simplify the
polynomial. b) Use Newton’s divided differences to write the Newton form of the interpolating polynomial. c) How many polynomials of degree 2 interpolate these 4 points? Explain. d) How many polynomials of degree 3 interpolate these 4 points? Explain. e) Find a polynomial of degree 100 that interpolates these 4 points. How many polynomials of
degree 100 interpolate these 4 points? Explain. \\ \\

	\begin{itemize}
		\item[a.)] 	We have x = $\{-2, 1, 1, 3\}$, y = $\{0,2,-3,1\}$. We can define P(x) = y$_1$L$_1$ + y$_2$L$_2$ + y$_3$L$_3$ + y$_4$L$_4$, where: \\ \\
		L$_1$ = $\cfrac{(x+1)(x-1)(x-3)}{(-1+2)(1+2)(3+2)}$ \\
		L$_2$ = $\cfrac{(x+2)(x-1)(x-3)}{(-2+1)(1+1)(3+1)}$ \\
		L$_3$ = $\cfrac{(x+2)(x+1)(x-3)}{(-1+-2)(-1+-1)(3-1)}$ \\
		L$_4$ = $\cfrac{(x+2)(x+1)(x-1)}{(-2+-3)(-1+-3)(1-3)}$ \\ \\
		
Thus our polynomial P(x) = $0 + 2L_2 + -3L_3 + 1L_4$ = $\frac{1}{40}(-21x^3 +18x^2 +121x + 2)$ \\ 
	\item[b.)] The table of coefficients is given below: \\
		{\renewcommand{\arraystretch}{2}%
	\begin{tabular}{lllll}
	-2 & 0 & - & - & - \\
	-1 & 2 & $\frac{2-0}{-1+2}$=2  & -  & -  \\
	-1 & 3 &$\frac{-3+-2}{2}$=$\frac{-5}{2}$ & $\frac{\frac{-5}{2} - 2}{1+2}$= $\frac{-3}{2}$  & - \\
	3 & 1 & $\frac{1+3}{3+1}$=2 & $\frac{2+\frac{5}{2}}{3+1}$=$\frac{9}{8}$  & $\frac{\frac{9}{8}+\frac{3}{2}}{5}$ = $\frac{21}{40}$
\end{tabular}}\quad \\ \\
Thus our polynomial P$_3$(x) =  $\frac{1}{40}(21x^3 - 18x^2 - 121x -42) + 1$\\
	\item[c.)] Due to the uniqueness theorem, since we have found a polynomial of degree 3 that interpolates these 4 points, there does not exist and polynomial of degree 2 that interpolates these 3 points.\\
	\item[d.)] Similarly due to the uniqueness theorem, we have found a polynomial of degree 3 that interpolates these 4 points, thus there exists a single polynomial of degree 1 that interpolates these points.\\
	\item[e.)] We can find a polynomial of degree 100 defined by $\frac{1}{40}(-21x^3 +18x^2 +121x + 2) + (x+2)(x+1)(x-1)(x-3)x^{96}$. There are an infinitely many polynomials of degree 100 that pass through the 4 points since we can multiple our $x^{100}$ polynomial by any constants and still obtain a polynomial that passes through the 4 points.
 	\end{itemize}
 	
 	\item How many degree 3 polynomials can intersect a given degree 7 polynomial at 8 points? Explain. \\ \\
 	According to the uniqueness interpolation theorem, there is only a single polynomial of degree $\leq$ 7 that passes through the 8 points, therefore it is impossible to have both a degree 3 and degree polynomial intersect at 8 points. \\
 	
 	\item $\#$4(a), p.163, Sauer, 3rd ed. ($\#$4(a), p.156, Sauer, 2nd ed.) Also, find an upper bound on the error for any x-value between 0 and 10. \\
 	The error function is given by: f(x) - P$_5$(x) = $\cfrac{(x-0)(x-2)(x-4)(x-6)(x-8)(x-10)}{6!}$f$^6$(c). \\
 	Our max of our function $|$f$^6$(c)$|$ = $\cfrac{6!}{(c+5)^7}$, which occurs at c=0 $\in$ [0,10] $\rightarrow$ $|$f$^6$(0)$|$ = $\cfrac{6!}{(5)^7}$
 	$|$f(1) - P(1)$|$ = $\cfrac{1}{5^7}$(1-0)(1-2)(1-4)(1-6)(1-8)(1-10) $\approx$ 0.012 \\ \\
 	Our max upper bound is given by: $\cfrac{x_n-x_1}{n!}M_n$ = $\cfrac{10-0}{1}*\cfrac{1}{5^7}$ = 0.000128
 	
 	\item Consider the Chebyshev polynomial T$_5$(x). a) Find this polynomial using the recursive definition, starting with T$_0$(x)=1 and T$_1$(x) = x.  List the Chebyshev interpolation nodes on [-4,2] for n = 5. c.) Find an upper bound on the absolute value of the interpolation error if f(x)=e$^-3x$ is being interpolated on these nodes. \\
	\begin{itemize}
		\item[a.)] 	Recursively...\\
 	T$_2$(x) = 2x(x)-1 =2x$^2$-1 \\
 	T$_3$(x) = 2x(2x$^2$-1) - x = 4x$^3$-3x \\
 	T$_4$(x) = 2x(4x$^3$-3x) - 2x$^2$-1 = 8x$^4$-8x$^2$+1 \\
 	T$_t$(x) = 2x(8x$^4$-8x$^2$ +1) - 4x$^3$-3x = 16x$^6$-20x$^3$+5x \\
 		\item[b.)] Our nodes are given by x$_i$ = $\cfrac{b+a}{2} + \cfrac{b-a}{2}cos(\cfrac{(2i - 1)\pi}{2n})$ $\rightarrow$ 3cos($\cfrac{(odd \ \pi)}{2n}$) - 1. \\
 		Nodes: 3cos($\frac{\pi}{10}$)-1, 3cos($\frac{3\pi}{10}$)-1, 3cos($\frac{5\pi}{10}$)-1, 3cos($\frac{7\pi}{10}$)-1, 3cos($\frac{9\pi}{10}$)-1 \\
 		\item[c.)] Our error is $\leq$ $\cfrac{(\frac{b+a}{2})^5}{n!2^{n-1}} * max(f^5(c))$. We know that the max(f$^6$(c)) on our interval [-4,2] is $|$-243e$^{12}|$ = 39549414, so our error becomes $\cfrac{(\frac{2+4}{2})^5}{5!2^4}39549414$ = 5005472.749. 	\\
	\end{itemize}
	
	\item Compute $\int^1_{-1}$ $\cfrac{[T_n(x)]^2}{\sqrt[2]{1-x^2}}$dx. \\ \\
	With change of base, we get $\int^\pi_{0}$ $\cfrac{cos^2(n\theta)}{\sqrt[2]{1-cos^2(x)}}(-sin(\theta))d\theta$ \\
	$-\int^\pi_{0}$ $cos^2(n\theta)d\theta$ $\rightarrow$ $-\int^\pi_{0}$ $\cfrac{1+cos(2n\theta)}{2}d\theta$ $\rightarrow$ $-\frac{1}{2}\int^\pi_{0}$ $1+cos(2n\theta)d\theta$ \\
	$-\frac{1}{2}\int^\pi_{0}$ $1+cos(2n\theta)d\theta$ = $-\cfrac{1}{2}$[$\pi$] = $\cfrac{-\pi}{2}$	\\
	
	\item $\#$8b, p. 184, Sauer, 3rd ed. ($\#$8b, p. 176, Sauer, 2nd ed.) \\ \\
	Given our points (0,0), (1,1), (2,6) \\
	$a_1 = y_1 = 0, a_2 = y_2 = 1$ \\
	$\delta_1$ = x$_2$-x$_1$ = 1, $\delta_2$ = x$_3$-x$_2$ = 1 \\
	$\Delta_1$ = y$_2$-y$_1$ = 1, $\Delta_2$ = y$_3$-y$_2$ = 5 \\ \\
	$\begin{bmatrix}
   1 & 0 & 0 \\
   1 & 4 & 0 \\
   0 & 0 & 1 \\
\end{bmatrix}$ 	$\begin{bmatrix}
   c_1  \\
   c_2  \\
   c_3  \\
\end{bmatrix}$ = $\begin{bmatrix}
   0  \\
   12  \\
   0  \\
\end{bmatrix}$ $\rightarrow$ $c_1 = c_3 = 0, c_2 = 3$ \\ \\
$d_1 = \cfrac{c_2 - c_1}{3\delta_1} = 1$, $d_2 = \cfrac{c_3 - c_2}{3\delta_2} = -1$\\
$b_1 = \cfrac{\Delta_1}{\delta_1} - \cfrac{\delta_1}{3}(2c_1+c_2) = 0$, $\cfrac{\Delta_2}{\delta_2} - \cfrac{\delta_2}{3}(2c_2+c_3) = 3$. \\ \\
Our splines are given by... \\
$S_1(x) = 0 + 0x + 0x^2 + x^3$ \\
$S_2(x) = 1 + 3x + 3x^3 - x^3$ \\ 
	
	\item A cubic spline is defined by [pic]. a) Find a, b, c, and d if there are natural end-conditions. b) Is the spline in part (a) parabolically terminated? Is the spline in part (a) not-a-knot? Explain. \\ \\
	
	
	$S_1(x_i) = S_2(x_i)$ for i=2,...,n-1 $\rightarrow$ $S_1(2) = S_2(2)$. \\
	$S_1(2) = 1+a-b$, $S_2(2)$ = $1$ $\rightarrow$ $b=a$. \\ \\
	
	$S^1_1(x_i) = S^1_2(x_i)$ for i=2,...,n-1 $\rightarrow$ $S^1_1(2) = S^1_2(2)$. \\
	$S^1_1(2) = a-3b$, $S^1_2(2)$ = c $\rightarrow$ $c = a-3b$. \\ \\
	
	$S^2_1(x_i) = S^2_2(x_i)$ for i=2,...,n-1 from natural end conditions. \\
	$S^2_1(x) = 0$, $S^2_2(x)$ = $\frac{-3}{2} + 6d$ $\rightarrow$ $d = \frac{1}{4}$. \\ 
	
	$S^2_1(x_i) = S^2_2(x_i)$ for i=2,...,n-1 $\rightarrow$ $S^2_1(2) = S^2_2(2)$. \\
	$S^2_1(2) = 0$, $S^2_2(2)$ = $-6b = \frac{-3}{2}$ $\rightarrow$ $b=a=\frac{1}{4}$. \\ \\
	$c = -2b = \frac{1}{2}$, Therefore $a = \cfrac{1}{4}, b = \cfrac{1}{4}, d = \cfrac{1}{4}, c = \cfrac{-1}{2}$. \\
	
	\item How many evenly-spaced knots (nodes), including the endpoints, are needed so that the absolute error between a linear spline and the function f(x)=cos(5x)it approximates on [-3, 4] is less than 10$^{-2}$ for any x. \\ \\
	The error is given by $|$E$_{n-1}(x)|$ $\leq$ $\cfrac{M_nh^n}{k_n}$ $\rightarrow$ $\cfrac{h^{n+1}}{4(n+1)}M_n$ $\leq$ $10^{-2}$, where $M_n$ = $MAX_{x \in [-3, 4]}(f^n(x))$ \\
	$M_n$ = $5^n$, so our error is $\cfrac{h^{n+1}}{4(n+1)}5^n$ $\leq$ $10^-2$. Solving for n, we get that we need at least 35 nodes to obtain an error $\leq$ $10^-2$.




\end{enumerate}
\end{document}