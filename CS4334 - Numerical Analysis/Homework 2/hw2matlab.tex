\documentclass[12pt]{article}

\usepackage[margin=1.0in]{geometry}
\usepackage{tabto}
\usepackage{amsmath}
\usepackage{framed}
\usepackage{listings}
\usepackage{color} %red, green, blue, yellow, cyan, magenta, black, white
\definecolor{mygreen}{RGB}{28,172,0} % color values Red, Green, Blue
\definecolor{mylilas}{RGB}{170,55,241}

\begin{document}

\title{CS / MATH 4334 : Numerical Analysis\\Homework Assignment 2}
\author{Matthew McMillian\\mgm160130@utdallas.edu}
\maketitle

\section*{MatLab Problems}

\lstset{language=Matlab,%
    %basicstyle=\color{red},
    breaklines=true,%
    morekeywords={matlab2tikz},
    morekeywords=[2]{1}, keywordstyle=[2]{\color{black}},
    identifierstyle=\color{black},%
    showstringspaces=false,%without this there will be a symbol in the places where there is a space
    numbers=left,%
    numberstyle={\tiny \color{black}},% size of the numbers
    numbersep=9pt, % this defines how far the numbers are from the text
    emph=[1]{for,end,break},emphstyle=[1]\color{red}, %some words to emphasise
    %emph=[2]{word1,word2}, emphstyle=[2]{style},    
}

\pagebreak

	\begin{enumerate}
	
	\item[] Problem 1 : ffalpos.m \noindent\rule{\textwidth}{1.0pt} \\
	\lstinputlisting{ffalpos.m}
	
	\item[] Problem 1 : falpos.m \noindent\rule{\textwidth}{1.0pt} \\
	\lstinputlisting{falpos.m}	
	
	$>>$ falpos.m 
	\begin{framed}
	cits =

     1.118146029604788e+00\\
     1.062353732855704e+00\\
     1.060497138511485e+00\\
     1.060437095603502e+00\\
     1.060435155605902e+00\\
     1.060435092926107e+00\\
     1.060435090900974e+00\\
     1.060435090835543e+00\\
     1.060435090833429e+00\\
     1.060435090833361e+00\\
     1.060435090833359e+00\\
     
    count =

    11
	\end{framed}
	
	\end{enumerate}
	
\pagebreak	
	
	\begin{enumerate}
	\item[] Problem 2 : newt1.m \noindent\rule{\textwidth}{1.0pt} \\
	\lstinputlisting{newt1.m}	
	
	\item[] Problem 2 : newt2.m \noindent\rule{\textwidth}{1.0pt} \\
	\lstinputlisting{newt2.m}	
	
		\item[] Problem 2 : fnewt.m \noindent\rule{\textwidth}{1.0pt} \\
	\lstinputlisting{fnewt.m}
	
		\item[] Problem 2 : fpnewt.m \noindent\rule{\textwidth}{1.0pt} \\
	\lstinputlisting{fpnewt.m}		

	$>>$ newt2.m 
	\begin{framed}
Iterates::

ans =

     1.767085782124950e+00\\
     1.538624098431182e+00\\
     1.315274752139667e+00\\
     1.098538133852653e+00\\
     8.915239987248459e-01\\
     6.999922093683941e-01\\
     5.329629816329148e-01\\
     4.012834512079965e-01\\
     3.132961399003459e-01\\
     2.696854759245819e-01\\
     2.582949829973730e-01\\
     2.575434526842937e-01\\
     2.575402738874424e-01\\
     2.575402738307063e-01\\

Error 1::

ans =

     1.318069672854064e-01\\
     1.126529257184075e+00\\
     1.143634792453416e+00\\
     1.161847033275790e+00\\
     1.176927760096816e+00\\
     1.178367340962222e+00\\
     1.145375182200481e+00\\
     1.047060053412795e+00\\
     8.558506476282967e-01\\
     5.757979815791652e-01\\
     2.727039364615275e-01\\
     6.617126557971360e-02\\
     4.229817522440887e-03\\

Error 2::

ans =

     1.318069672854064e-01\\
     8.546811146521263e+00\\
     7.702053241592080e+00\\
     6.841963471527189e+00\\
     5.965304965421256e+00\\
     5.074739283541284e+00\\
     4.186008531733245e+00\\
     3.340997594141816e+00\\
     2.608140388188166e+00\\
     2.050242044110255e+00\\
     1.686383195908350e+00\\
     1.500523816401236e+00\\
     1.449524776306588e+00\\

Since the derivative of the function is non-zero, we can determine that there are multiple roots. Also, we can determine that there is linear convergence since the error decreases at a somewhat linear rate.\\
Iterates:: 2.580000e-01\\
Multiplicity:: 4.692627e+05\\
Rate of Convergence:: 9.999979e-01\\
We have determined that we have a rate of convergence of approx: 1, therefore we have linear convergence for this newton's method.
	\end{framed}
	\end{enumerate}

	\pagebreak

	\begin{enumerate}
	\item[] Problem 3 : hw3.m \noindent\rule{\textwidth}{1.0pt} \\
	\lstinputlisting{hw3.m}	
	
	\pagebreak
	$>>$ hw3.m
	\begin{framed}
		fzero(function1, x) results::
(x, rootest, fval): 5, 1, 0\\
(Backward error, Forward error): 0, 0\\
(x, rootest, fval): 5.000000e+00, 1, 0\\
(Backward error, Forward error): 0, 0\\ 

fzero(function2, x) results::\\
(x, rootest, fval): 5, 9.999996e-01, 0\\
(Backward error, Forward error): 0, 4.312810e-07\\
(x, rootest, fval): 5.000000e+00, 1.000003e+00, 0\\
(Backward error, Forward error): 0, 3.095297e-06\\

Multiplicity f1: 1\\
Multiplicity f2: 3\\
As the multiplicity increases, stability decreases.
 Thus, with multiplicity 1 for the first function, we
 have a pretty stable algorithm. However, when we 
 introduce higher multiplicity in function 2, we start to 
 lose some of our stability and it begins to slightly affect 
 our approximations. The inital guess doesn't affect the first 
 function very much, but you can see the increased variablity in 
 the seconds function from the root difference.
	\end{framed}	
	
	\end{enumerate}

\end{document}