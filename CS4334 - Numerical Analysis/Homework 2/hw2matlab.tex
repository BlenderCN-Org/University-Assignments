\documentclass[12pt]{article}

\usepackage[margin=1.0in]{geometry}
\usepackage{tabto}
\usepackage{amsmath}
\usepackage{framed}
\usepackage{listings}
\usepackage{color} %red, green, blue, yellow, cyan, magenta, black, white
\definecolor{mygreen}{RGB}{28,172,0} % color values Red, Green, Blue
\definecolor{mylilas}{RGB}{170,55,241}

\begin{document}

\title{CS / MATH 4334 : Numerical Analysis\\Homework Assignment 2}
\author{Matthew McMillian\\mgm160130@utdallas.edu}
\maketitle

\section*{MatLab Problems}

\lstset{language=Matlab,%
    %basicstyle=\color{red},
    breaklines=true,%
    morekeywords={matlab2tikz},
    morekeywords=[2]{1}, keywordstyle=[2]{\color{black}},
    identifierstyle=\color{black},%
    showstringspaces=false,%without this there will be a symbol in the places where there is a space
    numbers=left,%
    numberstyle={\tiny \color{black}},% size of the numbers
    numbersep=9pt, % this defines how far the numbers are from the text
    emph=[1]{for,end,break},emphstyle=[1]\color{red}, %some words to emphasise
    %emph=[2]{word1,word2}, emphstyle=[2]{style},    
}

\pagebreak

	\begin{enumerate}
	
	\item[] Problem 1 : ffalpos.m \noindent\rule{\textwidth}{1.0pt} \\
	\lstinputlisting{ffalpos.m}
	
	\item[] Problem 1 : falpos.m \noindent\rule{\textwidth}{1.0pt} \\
	\lstinputlisting{falpos.m}	
	
	$>>$ falpos.m 
	\begin{framed}
	cits =

     1.118146029604788e+00\\
     1.062353732855704e+00\\
     1.060497138511485e+00\\
     1.060437095603502e+00\\
     1.060435155605902e+00\\
     1.060435092926107e+00\\
     1.060435090900974e+00\\
     1.060435090835543e+00\\
     1.060435090833429e+00\\
     1.060435090833361e+00\\
     1.060435090833359e+00\\
     
    count =

    11
	\end{framed}
	
	\end{enumerate}
	
\pagebreak	
	
	\begin{enumerate}
	\item[] Problem 2 : newt1.m \noindent\rule{\textwidth}{1.0pt} \\
	\lstinputlisting{newt1.m}	
	
	\item[] Problem 2 : newt2.m \noindent\rule{\textwidth}{1.0pt} \\
	\lstinputlisting{newt2.m}	

	$>>$ newt2.m 
	\begin{framed}
a)\\
  \tabto{0.5cm}1.a) Actual. value of pi from MatLab = 3.141593e+00\\
   \tabto{0.5cm}1.b) Approx. value of pi using Maclaurin series: 4arctan(1) = 3.141597e+00\\
   \tabto{0.5cm}2) Absolute Error: 4.140539e-06\\
   \tabto{0.5cm}3) Relative Error: 1.317974e-06\\
   \tabto{0.5cm}4) Number of 'k' terms needed to approx. in single percision: 16777216\\
b)\\
   \tabto{0.5cm}Eventually the next value in the series becomes extremely small (since k is constantly increasing in the denominator, the next value to be added in the series will be small). This is the result of the phenomenon SWAMPING, since we are trying to add two numbers whose sizes are very different (one large and one extremely small). Therefore, the percision will eventually lose track of the very small values computed due to the rounding and computational limitations.	
	\end{framed}
	\end{enumerate}

	\pagebreak

	\begin{enumerate}
	\item[] Problem 3 : hw3.m \noindent\rule{\textwidth}{1.0pt} \\
	\lstinputlisting{hw3.m}	
	
	\pagebreak
	$>>$ hw3.m
	\begin{framed}
		fzero(function1, x) results::
(x, rootest, fval): 5, 1, 0\\
(Backward error, Forward error): 0, 0\\
(x, rootest, fval): 5.000000e+00, 1, 0\\
(Backward error, Forward error): 0, 0\\ 

fzero(function2, x) results::\\
(x, rootest, fval): 5, 9.999996e-01, 0\\
(Backward error, Forward error): 0, 4.312810e-07\\
(x, rootest, fval): 5.000000e+00, 1.000003e+00, 0\\
(Backward error, Forward error): 0, 3.095297e-06\\

Multiplicity f1: 1\\
Multiplicity f2: 3\\
As the multiplicity increases, stability decreases.
 Thus, with multiplicity 1 for the first function, we
 have a pretty stable algorithm. However, when we 
 introduce higher multiplicity in function 2, we start to 
 lose some of our stability and it begins to slightly affect 
 our approximations. The inital guess doesn't affect the first 
 function very much, but you can see the increased variablity in 
 the seconds function from the root difference.
	\end{framed}	
	
	\end{enumerate}

\end{document}