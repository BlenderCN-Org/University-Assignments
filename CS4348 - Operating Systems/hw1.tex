\documentclass[12pt]{article}

\usepackage[margin=1.0in]{geometry}

\begin{document}

\title{CS4348 : Operating Systems Concepts\\Homework Assignment 1}
\author{Matthew McMillian\\mgm160130@utdallas.edu}
\maketitle

\begin{enumerate}
	
	\item Figure 1.11 on slide 26 is displaying what happens when an interrupt occurs 			within the OS, and how the system handles the memory and registers. In the figure, 			part (a) shows what happens when an interrupt happens an an arbitrary location 'n'. 		During this process, the current program counter + 1, general registers, and stack 			pointer get pushed to the control stack so that the task that triggered the interrupt 		doesn't overwrite anything from the inside of the task that the outside environment 		was using. Part (b) describes the events that occurs when the interrupted event ends. 		After the event ends, the program counter, general registers, and stack pointer get 		updated with the values stored on the control stack.

	\item Memory hierarchies are important for a system. Memory comes in many different 		sizes and speeds, and there are trade-offs with each one. Typically, faster memory is 		more expensive and it has less of a size, and slower memory is cheaper but it has a 		larger size. Therefore, in order to optimize our available options, we use both fast 		and slow memory in various sizes and speeds to get the most out of the system. A good 		example of this is the cache system and disk storage. For the most part, you access 		cached elements very frequently. For this reason you would need fast access speed 			which in turn means the cache must be smaller in size. With disk storage, you most 			likely won't access it very frequently, and thus you can forfeit your read/write speed 	for more storage space.

	\item Proven elsewhere about caches, if you fetch an instruction from a block there is 	a high probability that the next instruction will be in that same block. Similarly to 		the last question, if you read a small block into the cache, you will be able to read 		from it very fast instead of having to do another slow fetch from the memory. This 			combination of periodic slow fetching from disk to gain, in general, higher 				performance is the goal of the blocking size in caches. This size is very important; 		if you make the cache size is too large, you lose your performance, but if its too 			small you may not gain as much performance as you would like.

	\item A monitor program is somewhat like a queuing system. that is, it accepts a 			'batch' of jobs and runs them one at a time. In the 1940s - 1950s, people had to 			reserve times on terminals to run their code since machines were so expensive and 			slow, and it was a large inconvenience to developers. The monitor system was a 				solution to this since it allowed for a solid work flow as people were able to drop 		off their code to be put into a batch to be run, and they could come back later and 		get the results. In conclusion, the 	transition from one person at a time whom had 		to reserve time slots to the ability to drop 	off your code without having to wait 		allowed for fluidity within the enterprise environment.

	\item Memory paging is a way to load different pieces of some program, or \textit{pages}, (or partitions) of a program into the main memory without loading the entire program all at once.	This allows for an OS to run parts of multiple programs without loading all of them into the main memory. During this process, the main memory is removing pages that are not in use very often and loading in new pages that it will need more frequently. If a page is missing from memory, a \textit{page fault} occurs, and the new page is loaded into the memory. Without paging, we would run out of main memory space quickly and we wouldn't be able to run multiple programs with such efficiency as we can today. In a way, this facilitates virtual memory since it appears that we have access to more memory than we really do. We are using the same amount of memory for many program, instead of just limited it to only a single program.

	\item A microkernel architecture is quite different from a monolithic architecture. Instead of sticking all of the kernel requirements into a single huge, \textit{a.out} file, a microkernel works to minimize the amount functions that its' kernel uses. For the functions that the kernel does not use, the microkernel architecture uses other remote and local processes to handle these tasks. This approach increases the flexibility and scalibility of the architecture, however communication between the processes and the kernel, known as \textit{interprocess communication (IPC)} could be inherently slower than direct function calls from the same file, resulting in poorer performance in some cases.

\end{enumerate}
	

\end{document}
